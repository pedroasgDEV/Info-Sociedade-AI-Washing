%------------ Design dos slides -------------

\documentclass{beamer}
\usetheme{Warsaw}
\useoutertheme{miniframes}
\usecolortheme{beaver}

%------------ Language packages -------------

\usepackage[utf8]{inputenc}
\usepackage[T1]{fontenc}
\usepackage[brazil]{babel}

%------------- Useful packages --------------

\usepackage{booktabs}
\usepackage{graphicx}
\usepackage{adjustbox}
\usepackage{hyperref}
\usepackage{csquotes}

%----------------- bibtex --------------------

\usepackage[backend=bibtex,style=authoryear]{biblatex}
\addbibresource{main.bib} 

%------------------ TITLE ------------------- 

\title{Aplicação da IA como elemento de marketing}
\subtitle{Entre a inovação e o mau uso no mercado e na academia}
\author{Pedro Augusto Sousa Gonçalves \\ Bruno \\ Tiago}
\institute{Departamento de Computação \\ Universidade Federal de Ouro Preto}
\date{22 de outubro de 2025}

%---------------- Document ------------------ 

\begin{document}

\begin{frame}
    \titlepage
\end{frame}

% 2. Slide do Sumário
\begin{frame}
    \frametitle{Sumário}
    \tableofcontents
\end{frame}

\section{Motivação} 

\begin{frame}{AI Washing}
    \begin{columns}
        \begin{column}{0.60\textwidth}
            \begin{itemize}
                \item Nossa análise foca no \textbf{AI Washing}: o uso da IA como selo de marketing, e não como uma ferramenta de valor real.
                
                \item Tal como foi exposto em \cite{Woollacott2024AIWashing}, atualmente a IA tornou-se um selo de inovação para agregar valor percebido, favorecendo o uso \textbf{cosmético} e \textbf{impensado}, focado mais no hype do que no valor real.
                
            \end{itemize}
        \end{column}
        \begin{column}{0.40\textwidth}
            \begin{figure}
                \includegraphics[width=1.1\textwidth]{meme1.png}
            \end{figure}
        \end{column}
    \end{columns}

    \href{https://youtu.be/s4wlrxFf2lM?si=Bklwu49XH0oW3PKs&t=2276}{Vídeo do Linus Torvalds falando sobre o hype da IA}
\end{frame}


\begin{frame}{Exemplos}
    \begin{itemize}
        \item \textbf{Humanos Disfarçados de IA}: A tecnologia não está pronta, então a empresa usa humanos "nos bastidores" para executar as tarefas, fingindo ser um algoritmo avançado.
        \item \textbf{Automação Simples (Se-Então)}: A empresa substitui seu antigo sistema de FAQ por um chatbot, mas a lógica por trás dele não é inteligência e sim automação, mas é vendido como IA.
        \item \textbf{Estatística Reembalada}: Ocorre quando técnicas estatísticas padrão (como regressão logística ou keyword matching) são vendidas como "IA" ou "Machine Learning".
        \item \textbf{Over-Engineering}: Ocorre quando a solução mais simples e transparente é descartada em favor da IA, apenas para que a empresa ou o pesquisador possa alegar que está usando "tecnologia de ponta".
    \end{itemize}
\end{frame}

\begin{frame}{Consequências}
    \begin{itemize}
        \item \textbf{Alocação Incorreta de Capital (A Bolha da IA)}: Investidores têm dificuldade em distinguir o hype da realidade.
        \begin{figure}
                \includegraphics[width=.6\textwidth]{noticia.png}
                \caption{\cite{thurrott_openai_loss2025}}
        \end{figure}
        \item \textbf{Desperdício de Recursos e Dívida Técnica}: Gasta-se tempo de engenheiros de elite, poder computacional e energia elétrica para resolver problemas simples.
        \item \textbf{Poluição Científica}: Na academia, o uso impensado enche os periódicos científicos com pesquisas de baixo rigor.
    \end{itemize}
\end{frame}

\section{Introdução} 

\begin{frame}{O que é IA, de Fato?}
    \begin{itemize}
        \item A inteligência artificial envolve o desenvolvimento de sistemas que realizam tarefas antes restritas ao raciocínio humano (\cite{russell_norvig_aima2021}).
        \item \textbf{IA Forte}: É a IA da ficção científica. Uma máquina com consciência, capaz de aprender e executar qualquer tarefa intelectual que um humano pode. Isto não existe.
        \item \textbf{IA Fraca}: É um sistema treinado para executar uma tarefa específica de forma muito eficaz (ex: reconhecer um rosto, traduzir um texto, jogar xadrez, dirigir um carro).
        \item O "AI Washing" muitas vezes insinua ter capacidades de IA Forte, quando, na melhor das hipóteses, entrega uma IA Fraca ou, na pior, nada.
    \end{itemize}
\end{frame}

\begin{frame}{O que é IA, de Fato?}
    \begin{columns}
        \begin{column}{0.60\textwidth}
            \begin{figure}
                \includegraphics[width=1\textwidth]{gpt_calc.png}
            \end{figure}
        \end{column}
        \begin{column}{0.40\textwidth}
            \begin{figure}
                \includegraphics[width=0.6\textwidth]{gpt_coding.jpeg}
            \end{figure}
        \end{column}
    \end{columns}
\end{frame}

\begin{frame}{A História da IA}
    \begin{itemize}
        \item \textbf{Anos 50}
        \begin{itemize}
            \item \textbf{Alan Turing (1950)}: Propõe o "Teste de Turing".
            \item \textbf{Conferência de Dartmouth (1956)}: O termo "Inteligência Artificial" é oficialmente cunhado. Foram feitas promessas extraordinárias para o futuro da IA Forte, que foram financiadas por agências governamentais.
        \end{itemize}
        \item \textbf{O 1º "Inverno da IA" (Anos 70-80)}
        \begin{itemize}
            \item \textbf{Explosão Combinatória}: Os pesquisadores descobriram que os seus programas funcionavam bem em problemas fictícios (\textbf{controlados}), mas falhavam no mundo real (\textbf{complexo demais}).
            \item \textbf{Hardware e Dados Insuficientes}: Os computadores eram lentos, caros e tinham memória minúscula.
            \item \textbf{O Relatório de \cite{lighthill1973}}: Demonstrou que as pesquisas em IA falharam em cumprir suas promessas. Em resposta, o financiamento foi cortado.
        \end{itemize}
    \end{itemize}
\end{frame}

\begin{frame}{A História da IA}
    \begin{itemize}
        \item \textbf{2º "Inverno da IA" (Anos 90) (\cite{crevier1993_ai_history})}
        \begin{itemize}
            \item \textbf{IA Fraca como a "Nova Promessa"}: A nova ideia era capturar o conhecimento de um especialista humano e colocá-lo num programa. 
            \item \textbf{Sistema Especialista}: Um motor de inferência gigante com milhares de regras IF-THEN.
            \item \textbf{Máquinas Lisp}: O Hype Comercial fez as empresas investiram bilhões de dólares. Surgiram empresas especializadas em computadores caríssimos feitos só para rodar IA.
            \item \textbf{Colapso}
            \begin{itemize}
                \item O "Gargalo da Aquisição de Conhecimento".
                \item Os sistemas eram "frágeis".
                \item Alto Custo e Manutenção.
            \end{itemize}
        \end{itemize}
    \end{itemize}
\end{frame}

\begin{frame}{A História da IA}
    \begin{itemize}
        \item \textbf{O "Big Bang" (Pós-2010) ou "Verão da IA"}
        \begin{itemize}
            \item \textbf{ImageNet (\cite{ImageNet})}: Foi um esforço acadêmico massivo para criar uma base de dados gigantesca de milhões de imagens.
            \item \textbf{Hardware (GPUs e a NVIDIA)}:  Foi percebido que podiam usar GPUs de videojogos para acelerar o treino de IA em 100x ou 1000x. O que demoraria um ano, agora demorava dias.
            \item \textbf{Deep Learning (\cite{lecun_bengio_hinton_2015})}: Pesquisadores como Geoffrey Hinton, Yann LeCun e Yoshua Bengio refinaram os algoritmos e provaram que redes neurais profundas poderiam funcionar se tivessem dados e poder computacional suficientes (\textbf{ganharam o nobel da física em 2024}).
        \end{itemize}
        \item A história mostra que a IA é um campo cíclico de hype e desilusão. O "AI Washing" é o sintoma comercial do pico de hype deste "Verão" atual.
    \end{itemize}
\end{frame}

\begin{frame}{Quais Tecnologias Estão por Trás do "IA" Hoje?}
    \begin{itemize}
        \item \textbf{Machine Learning (\cite{abdelkarim_pfeuffer_hinz_2021})}: Modelos estatísticos que encontram padrões e fazem previsões.
        \item \begin{itemize}
            \item Regressão Logística, Árvores de Decisão, SVMs. 
            \item Prever se um cliente vai desistir, decidir se um e-mail é spam, segmentar clientes para marketing.
        \end{itemize}
        \item \textbf{Deep Learning}: É um subcampo do Machine Learning que usa "Redes Neurais Artificiais" com muitas camadas.
        \begin{itemize}
            \item \textbf{CNNs (Convolutional Neural Networks)}: Usam filtros matemáticos para aprender a detetar arestas, texturas, formas e, finalmente, objetos..
            \item \textbf{GANs (Generative Adversarial Networks)}:um Gerador que tenta criar imagens realistas, e um Discriminador que tenta apanhar as falsificações.
            \item \textbf{Transformers (\cite{vaswani2017_attention})}:um Gerador que tenta criar imagens realistas, e um Discriminador que tenta apanhar as falsificações.
        \end{itemize}
    \end{itemize}
\end{frame}

\begin{frame}{Quais Tecnologias Estão por Trás do "IA" Hoje?}
    \begin{itemize}
        \item \textbf{LLMs (Large Language Models) (\cite{LLM})}: São o tipo mais avançado de IA Generativa, focados em processar e gerar texto em grande escala usando a arquitetura \textbf{Transformer}.
        \begin{itemize}
            \item \textbf{Tarefa Principal}: O treino de um LLM baseia-se num objetivo surpreendentemente simples: "prever a próxima palavra".
            \item \textbf{Processo}: O modelo é alimentado com uma quantidade massiva de texto (quase toda a Internet: Wikipedia, livros, artigos, código-fonte). Ele recebe uma frase e tenta prever a palavra seguinte.
            \item \textbf{Aprendizagem}: Ele é forçado a aprender as regras subjacentes do mundo contidas na linguagem, tais como: gramática, sintaxe, fatos, lógica, raciocínio, capacidade de programação.
        \end{itemize}
        \item As LLMs são o maior avanço da IA até agora, levando muitos a crer que atingimos a IA Forte, o que alimenta um hype massivo que, por sua vez, intensifica o "AI Washing".
    \end{itemize}
\end{frame}

\section{Academia} 

\section{Mercado}

\section{Conclusão}

\begin{frame}{Medo como Motor de Hype em IA}
    \begin{itemize}
        \item As Big-tec buscando refugio no medo coletivo.
        \item Narrativas de risco extremo em IA criam urgência, atenção midiática e corrida tecnológica.
        \item O medo de IAs superpoderosas aumenta a percepção de valor da tecnologia e atrai investimentos acelerados.
        \item Projetos como AI 2027 (\cite{AI2027}) usam cenários futurísticos que reforçam essa sensação de iminência, estimulando empresas e governos a investir mais rápido.
        \item O “terror tecnológico” funciona como um catalisador econômico: quanto maior o medo, maior o interesse.
    \end{itemize}
\end{frame}

\begin{frame}{Medo como Motor de Hype em IA - I Have No Mouth, and I Must Scream}
    \begin{columns}
        \begin{column}{0.7 \textwidth}
            \begin{itemize}
                \item O conto de \cite{Ellison1967} apresenta uma IA absoluta e cruel, que simboliza a perda total de agência humana.
                \item Esse imaginário influencia o debate atual ao reforçar medos de máquinas incontroláveis.
                \item Metaforicamente, o final, “não tenho boca e preciso gritar”, representa o medo da impotência humana frente a forças tecnológicas e a do retorno ao primitivo (ex: "A IA vai roubar seu emprego", "Não temos controle dá IA" e etc...).
            \end{itemize}
        \end{column}
        \begin{column}{0.3 \textwidth}
            \begin{figure}
                \includegraphics[width=1\textwidth]{IhaveNo.jpg}
            \end{figure}
        \end{column}
    \end{columns}
\end{frame}

\begin{frame}{Medo como Motor de Hype em IA - Impactos Negativos}
    \begin{itemize}
        \item Produz fantasmagorias tecnológicas, onde hipóteses remotas passam a parecer inevitáveis.
        \item A estratégia não busca apenas convencer, mas ocupar o imaginário. Ao cultivar cenários extremos, as empresas desviam o olhar das limitações técnicas e mantêm ativa a expectativa de ruptura iminente.
        \item Nesse terreno fértil, aprofunda o AI washing, inflar artificialmente a capacidade de inteligência, autonomia e poder de sistemas limitados para sustentar confiança e capital. É a estética do “quase mágico” que disfarça engenharia convencional, tornando-se quase mistico, nebuloso.
        \item Esse movimento revela um desespero estrutural das empresas para manter a bolha de IA.
    \end{itemize}
\end{frame}

\begin{frame}{A Síntese do Hype e o Custo Real}
    \begin{itemize}
        \item \textbf{O Ciclo da IA}: Vimos que a história da IA é um ciclo de hype e "Invernos". O "Verão" atual, impulsionado pelo "Big Bang" do Deep Learning e o poder dos LLMs, criou o maior hype já visto.
        \item \textbf{O "AI Washing"}: Vimos que a história da IA é um ciclo de hype e "Invernos". O "Verão" atual, impulsionado pelo "Big Bang" do Deep Learning e o poder dos LLMs, criou o maior hype já visto.
        \item \textbf{O Custo Real}: Vimos que este "mau uso" impensado tem consequências reais.
        \begin{itemize}
            \item \textbf{No Mercado}: Cria concorrência desleal e uma bolha de capital.
            \item \textbf{Na Academia}: Gera "poluição científica".
            \item \textbf{No Planeta}: Desperdiça recursos naturais, principalmente, elétricos e hídricos para sustentar uma fachada.
        \end{itemize}
    \end{itemize}
\end{frame}

\begin{frame}{Como Mitigar o "AI-Washing"?}
    \begin{itemize}
        \item \textbf{Na Academia:}: Focar na reprodutibilidade e interpretabilidade (\cite{lipton2018_mythos_interpretability}), em vez de recompensar o uso impensado de "caixas pretas" apenas por serem novidade.
        \item \textbf{No Mercado}: Focar no valor tangível da IA, não no hype.
        \item \textbf{Na Regulação}: Reguladores devem definir e punir o "AI Washing" (\cite{Woollacott2024AIWashing}) para proteger consumidores e investidores.
        \item \textbf{Em Nós (Desenvolvedores)}: Alfabetização crítica e principalmente ...
    \end{itemize}
\end{frame}

\begin{frame}{Como Mitigar o "AI-Washing"?}
    \begin{figure}
        \includegraphics[width=1\textwidth]{kiss.png}
    \end{figure}
\end{frame}

\begin{frame}[allowframebreaks]{Referências}
    \printbibliography
\end{frame}

\end{document}
